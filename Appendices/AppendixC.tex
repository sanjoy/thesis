\chapter{Integer addition and bit flips}

\label{AppendixC}

\lhead{Appendix C. \emph{Integer addition and bit flips}}

The hamming distance between an integer and the integer incremented by
some amount is an important factor in determining data-address bus
switching costs associated with a program accessing array elements.
Here we show that adding integers $A$ and $B$ with bit-widths $W_A$
and $W_B$ respectively incurs $\frac{min(W_A, W_B)}{2} + 1$ bit
\textit{flips} in the result on an average.  Formally

\begin{equation}
  \label{eq:master-flip}
  E[\eta(A, A \uplus B) \mid A \in \{0, 1\}^{W_A}, B \in \{0,
    1\}^{W_B} ] \approx \frac{min(W_A, W_B)}{2} + 1
\end{equation}

where $\eta : (\{0, 1\}^*, \{0, 1\}^*) \to \mathbb{Z}$ computes the
hamming distance between its arguments and $\uplus : (\{0, 1\}^*, \{0,
1\}^*) \to \{0, 1\}^*$ is binary addition.  As an example, with $A =
(0, 1, 1, 0, 1)$ to $B = (1, 0, 0, 0, 1)$, $A \uplus B = (1, 1, 1, 0,
1)$ and $\eta(A, A \uplus B) = 2$.

Without loss of generality we assume $W_A \geq W_B$, making $min(W_A,
W_B) = W_B = N$.  Let $P_c(n)$ denote the probability of a non-zero
carry bit from adding the bits at location $n$ (including a previous
carry bit, if any) and $P_f(n)$ denote the probability of the $n^{th}$
bit flipping in $A$ after the addition.

By considering all the eight possibilities, it can be seen that for $n
< N$

\begin{equation}
    P_f(n) = \overbrace{\frac{P_c(n - 1)}{2}}^{\text{With carry}} +
    \underbrace{\frac{1 - P_c(n - 1)}{2}}_{\text{Without carry}} = \frac{1}{2}
\end{equation}

To calculate $P_f(n)$ for $n \geq N$, we first calculate $P_c(n)$ for
$n \geq N$.  Letting $C_i$ denote the $i^{th}$ carry bit (such that
$P_c(n) = Pr[C_n = 1]$), we observe

\begin{equation}
  P_c(N - 1) = Pr[C_{N - 1} = 1] = Pr[\nu(A\sembrack{N} \uplus
    B\sembrack{N}) > 2^N - 1]
\end{equation}

where $X\sembrack{Y}$ denotes the lower $Y$ bits of $X$ and $\nu :
\{0, 1\}^* \to \mathbb{Z}$ maps a binary string to its canonical integral
value, interpreted as an integer represented in base 2.

Conditioning on $B\sembrack{N}$ we get

\begin{eqnarray*}
  \label{eq:approximation-before-N}
  P_c(N - 1) & = & \displaystyle\sum\limits_{k=0}^{2^{N} - 1}
  Pr[\nu(A\sembrack{N} \uplus B\sembrack{N}) > 2^N - 1 \mid \nu(B\sembrack{N}) = k]
  \times Pr[\nu(B\sembrack{N}) = k] \\
  & = & \displaystyle\sum\limits_{k=0}^{2^{N} - 1}
  Pr[\nu(A\sembrack{N}) > 2^N - 1 - k \mid \nu(B\sembrack{N}) = k]
  \times Pr[\nu(B\sembrack{N}) = k] \\
  & = & \displaystyle\sum\limits_{k=0}^{2^{N} - 1}
  \left(\frac{k}{2^N}\right) \times
  \left(\frac{1}{2^N}\right) \\
  & = & \displaystyle\sum\limits_{k=0}^{2^{N} - 1} \frac{k}{2^{2 N}} \\
  & = & \varphi(N)
\end{eqnarray*}

Moreover, inductively

\begin{equation}
  P_c(N - 1 + t) =
  \begin{cases}
    \varphi(N)  & t = 0 \\
    \\
    \frac{P_c(N - 1 + (t - 1))}{2} & t > 0
  \end{cases}
\end{equation}

implying

\begin{equation}
  P_c(N - 1 + t) = \frac{\varphi(N)}{2^t}
\end{equation}

We observe that $P_f(n) = P_C(n - 1)$ for $n \geq N$, since $B$ does
not contribute anything to the $N^{th}$ bit and above, giving us


\begin{eqnarray*}
  P_f(n) & = &
  \begin{cases}
    \frac{1}{2} & n < N \\
    \\
    \frac{\varphi(N)}{2^{n - N}} & n \geq N \\
  \end{cases} \\
\end{eqnarray*}


The expected hamming distance between $A$ and $A \uplus B$ is therefore

\begin{equation}
  \label{eq:no-assumptions}
  E[\eta(A, A \uplus B)]
  = \displaystyle\sum\limits_{k=0}^{W_A - 1} P_f(k) \\
  = \displaystyle\sum\limits_{k=0}^{N - 1} \frac{1}{2} + 
  \displaystyle\sum\limits_{k=N}^{W_A - 1} \frac{\varphi(N)}{2^{k - N}}
\end{equation}

We now look at some specific cases

\begin{enumerate}
  \item $2^N \gg 0$ (in practice, $N \geq 6$ is enough)

    \begin{equation}
      \varphi(N) = \displaystyle\sum\limits_{k=0}^{2^{N} - 1} \frac{k}{2^{2 N}} = 
      \frac{1}{2} \frac{2^N (2^N - 1)}{(2^N) (2^N)}
      \approx \frac{1}{2}
    \end{equation}

  \item $W_A \gg W_b \equiv W_A \gg N$

    \begin{equation}
      \label{eq:approx-large-wa}
      \displaystyle\sum\limits_{k=N}^{W_A - 1} \frac{\varphi(N)}{2^{k - N}} \approx
      \displaystyle\sum\limits_{k=0}^{\infty} \frac{\varphi(N)}{2^k} = 2 \times \varphi(N)
    \end{equation}

  \item $W_A \gg W_b \equiv W_A \gg N$ \textit{and} $2^N \gg 0$

    \begin{equation}
      E[\eta(A, A \uplus B)] \approx \frac{N}{2} + 2 \times \varphi(N) = 
      \frac{min(W_A, W_B)}{2} + 1
    \end{equation}

  \item $B$ is known to be 1.

    We first find $E[\eta(A, A \uplus B)]$ given $W_B = 1$.  For this,
    $\varphi(N)$ can be calculated to be $\frac{1}{4}$.  Since
    approximation~\ref{eq:approx-large-wa} still applies, using
    equation~\ref{eq:no-assumptions} we have our expectation as 1.

    Now, conditioning on $B$, we get

    \begin{eqnarray*}
      E[\eta(A, A \uplus B)] & = & E[\eta(A, A \uplus B) \mid B = 0]
      \times Pr[B = 0] + \\
      && E[\eta(A, A \uplus B) \mid B = 1] \times Pr[B
        = 1]
    \end{eqnarray*}

    giving us

    \begin{equation}
      \label{eq:average-flips}
      E[\eta(A, A \uplus (1))] \approx 2
    \end{equation}

    An alternate, less general proof of the same theorem can be given
    as follows

    \begin{adjustwidth}{2.5em}{0pt}

    Let $F_k$ be the total number of bit flips encountered when
    incrementing 0 to $2^k - 1$ in steps of 1.  For instance, $F_2$ is 4
    since we encounter 1 bit flip going from 00 to 01, 2 when going from
    01 to 10 and 1 more when going from 10 to 11.

    Since going from 0 to $2^k - 1$ involves going from 0 to $2^{k - 1} -
    1$ \textit{twice} in the lower $k - 1$ bits and doing one $k$ bit flip
    when going from $2^{k - 1} - 1$ to $2^{k - 1}$


    \begin{equation} \label{eq:recurrence}
      F_k = 2 \times F_{k - 1} + k
    \end{equation}

    Solving the recurrence \ref{eq:recurrence} gives us

    \begin{equation} \label{eq:solved}
      F_k = 2^{k + 1} - (k + 2)
    \end{equation}

    The average number of bit flips when incrementing a $k$-bit integer is
    therefore $\frac{F_k}{2^k} \approx 2$

    \end{adjustwidth}

\end{enumerate}

Note that with $W_A = W_B$, $W_A \gg W_B$ is no longer valid and
equation~\ref{eq:no-assumptions} degenerates into

\begin{equation}
  E[\eta(A, A \uplus B)] = \displaystyle\sum\limits_{k=0}^{N - 1} \frac{1}{2} =
  \frac{N}{2}
\end{equation}

This is expected -- when $W_A = W_B = N$, $E[\eta(A, A \uplus B)] =
E[\eta(A, B)]$ since given $A$, $A \uplus B$ has the same distribution as
$B$.  $\eta$ distributes over the individual bits (which are
independently distributed) giving $E[\eta(A, B)] = N \times E[\eta(P,
  Q) \mid P, Q \in \{0, 1\}] = \frac{N}{2}$.
 
