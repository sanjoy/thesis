\chapter{Integer addition and bit flips}

\label{AppendixC}

\lhead{Appendix C. \emph{Integer addition and bit flips}}

The hamming distance between an integer and the integer incremented by
some amount is an important factor since it determines the switching
costs associated with a program accessing array elements.  Here we
show that adding integers $A$ and $B$ incurs a $\frac{min(W_a,
  W_b)}{2} + 1$ bit \textit{flips} on an average, where $W_x$ is the
width of $x$ in bits.  As an example, adding $A = 01101b$ to $B =
10001b$ \textit{flips} the most and the least significant bits of $A$,
making the hamming distance between $A$ and $A + B$ equal to 2.

Let $A$ and $B$ be two machine integers with $W_a$ the width of $A$ in
bits and $W_b$ the width of $B$ in bits and let $N = min(W_a, W_b)$.
Let $P_c(n)$ be the probability of an a carry bit from adding the bits
at location $n$ (including a previous carry bit, if any).  Without
loss of generality, we assume $W_A \geq W_b$.  Let $P_f(n)$ be the
probability of the $n^{th}$ bit flipping in $A$ after the addition.

By considering all the eight possibilities, it can be seen that when
$n < N$

\begin{equation}
    P_f(n) = \frac{P_c(n - 1)}{2} + \frac{1 - P_c(n - 1)}{2} =
    \frac{1}{2}
\end{equation}

To calculate $P_f(n)$ for $n \geq N$, we first calculate $P_c(n)$ for
$n \geq N$.  Letting $C_i$ denote the $i^{th}$ carry bit, we observe

\begin{equation}
  P_c(N - 1) = Pr[C_{N - 1} = 1] = Pr[A\sembrack{N} + B\sembrack{N} >
    2^N - 1]
\end{equation}

where $X\sembrack{Y}$ is the integer we get considering only the
lowest $Y$ bits of $X$.

Conditioning on $B\sembrack{N}$

\begin{eqnarray*}
  P_c(N) & = & \displaystyle\sum\limits_{k=0}^{2^{N} - 1}
  Pr[A\sembrack{N} > 2^N - k \mid B\sembrack{N} = k] Pr[B\sembrack{N} = k] \\
  & = & \displaystyle\sum\limits_{k=0}^{2^{N} - 1} \frac{k}{(2^N - 1)^2} \\
  & \approx & \frac{1}{2}
\end{eqnarray*}

We observe that $P_f(n) = P_C(n - 1)$ for $n \geq N$, since $B$ does
not contribute anything to the $N^{th}$ bit and above.  Moreover,
inductively

\begin{eqnarray*}
  P_c(N + t) & = &
  \begin{cases}
    \frac{1}{2} & t = 0 \\
    \\
    \frac{P_c(N + t - 1)}{2} & t > 0
  \end{cases} \\
  & = & \frac{1}{2^t}
\end{eqnarray*}

Thus, the expected hamming distance between $A$ and $A + B$ is

\begin{equation}
  E[Hamming(A, A + B)] = \displaystyle\sum\limits_{k=0}^{N - 1} \frac{1}{2} + 
  \displaystyle\sum\limits_{k=N}^{W_A} \frac{1}{2^{k - N}} + 
\end{equation}

When $W_A \gg W_b$, 

\begin{equation}
  \displaystyle\sum\limits_{k=N}^{W_A} \frac{1}{2^{k - N}} = 
  \displaystyle\sum\limits_{k=W_B}^{W_A} \frac{1}{2^{k - W_B}} \approx
  \displaystyle\sum\limits_{k=0}^{\infty} \frac{1}{2^k} = 1
\end{equation}

making

\begin{equation}
  E[Hamming(A, A + B)] \approx \frac{N}{2} + 1 = 
  \frac{min(W_a, W_b)}{2} + 1
\end{equation}

For an increment by 1, we have $W_B = 1$ and $W_A$ equal to the
machine integer width.  Thus, we have

\begin{equation}
  \label{proof:average-2-flips}
  E[Hamming(A, A + 1)] \approx \frac{1}{2} + 1 = \frac{3}{2}
\end{equation}
