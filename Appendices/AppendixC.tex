\chapter{Integer addition and bit flips}

\label{AppendixC}

\lhead{Appendix C. \emph{Integer addition and bit flips}}

The hamming distance between an integer and the integer incremented by
some amount is an important factor in determining data-address bus
switching costs associated with a program accessing array elements.
Here we show that adding integers $A$ and $B$ with bit-widths $W_A$
and $W_B$ respectively incurs $\frac{min(W_A, W_B)}{2} + 1$ bit
\textit{flips} in the result on an average.  Formally

\begin{equation}
  E[\eta(A, A \uplus B) \mid A \in \{0, 1\}^{W_A}, B \in \{0,
    1\}^{W_B} ] \approx \frac{min(W_A, W_B)}{2} + 1
\end{equation}

where $\eta : (\{0, 1\}^*, \{0, 1\}^*) \to \mathbb{Z}$ computes the
hamming distance between its arguments and $\uplus : (\{0, 1\}^*, \{0,
1\}^*) \to \{0, 1\}^*$ is binary addition.  As an example, with $A =
(0, 1, 1, 0, 1)$ to $B = (1, 0, 0, 0, 1)$, $A \uplus B = (1, 1, 1, 0,
1)$ and $\eta(A, A \uplus B) = 2$.

Without loss of generality we assume $W_A \geq W_B$, making $min(W_A,
W_B) = W_A = N$.  Let $P_c(n)$ denote the probability of a non-zero
carry bit from adding the bits at location $n$ (including a previous
carry bit, if any) and $P_f(n)$ denote the probability of the $n^{th}$
bit flipping in $A$ after the addition.

By considering all the eight possibilities, it can be seen that for $n
< N$

\begin{equation}
    P_f(n) = \overbrace{\frac{P_c(n - 1)}{2}}^{\text{With carry}} +
    \underbrace{\frac{1 - P_c(n - 1)}{2}}_{\text{Without carry}} = \frac{1}{2}
\end{equation}

To calculate $P_f(n)$ for $n \geq N$, we first calculate $P_c(n)$ for
$n \geq N$.  Letting $C_i$ denote the $i^{th}$ carry bit (such that
$P_c(n) = Pr[C_n = 1]$), we observe

\begin{equation}
  P_c(N - 1) = Pr[C_{N - 1} = 1] = Pr[\nu(A\sembrack{N} \uplus
    B\sembrack{N}) > 2^N - 1]
\end{equation}

where $X\sembrack{Y}$ denotes the lower $Y$ bits of $X$ and $\nu :
\{0, 1\}^* \to \mathbb{Z}$ maps a binary string to its canonical integral
value, interpreted as an integer represented in base 2.

Conditioning on $B\sembrack{N}$ we get

\begin{eqnarray*}
  \label{eq:approximation-before-N}
  P_c(N) & = & \displaystyle\sum\limits_{k=0}^{2^{N} - 1}
  Pr[\nu(A\sembrack{N}) > 2^N - k \mid \nu(B\sembrack{N}) = k]
  Pr[\nu(B\sembrack{N}) = k] \\
  & = & \displaystyle\sum\limits_{k=0}^{2^{N} - 1} \frac{k}{(2^N - 1)^2} \\
  & \approx & \frac{1}{2}
\end{eqnarray*}

We observe that $P_f(n) = P_C(n - 1)$ for $n \geq N$, since $B$ does
not contribute anything to the $N^{th}$ bit and above.  Moreover,
inductively

\begin{eqnarray*}
  P_c(N + t) & = &
  \begin{cases}
    \frac{1}{2} & t = 0 \\
    \\
    \frac{P_c(N + t - 1)}{2} & t > 0
  \end{cases} \\
  & = & \frac{1}{2^{t + 1}}
\end{eqnarray*}

Thus, the expected hamming distance between $A$ and $A \uplus B$ is

\begin{equation}
  \label{eq:no-assumptions}
  E[\eta(A, A \uplus B)] = \displaystyle\sum\limits_{k=0}^{N - 1} \frac{1}{2} + 
  \displaystyle\sum\limits_{k=N}^{W_A} \frac{1}{2^{k + 1 - N}}
\end{equation}

When $W_A \gg W_b \equiv W_A \gg N$, 

\begin{equation}
  \displaystyle\sum\limits_{k=N}^{W_A} \frac{1}{2^{k + 1 - N}} \approx
  \displaystyle\sum\limits_{k=0}^{\infty} \frac{1}{2^{k + 1}} = 1
\end{equation}

making

\begin{equation}
  E[\eta(A, A \uplus B)] \approx \frac{N}{2} + 1 = 
  \frac{min(W_A, W_B)}{2} + 1
\end{equation}

For an increment by 1, we have $W_B = 1$ and $W_A$ equal to the
machine integer width.  Thus

\begin{equation}
  \label{proof:average-2-flips}
  E[\eta(A, A \uplus (1))] \approx \frac{1}{2} + 1 = \frac{3}{2}
\end{equation}

Note that with $W_A = W_B$, $W_A \gg W_B$ is no longer valid and
equation~\ref{eq:no-assumptions} degenerates into

\begin{equation}
  E[\eta(A, A \uplus B)] = \displaystyle\sum\limits_{k=0}^{N - 1} \frac{1}{2} =
  \frac{N}{2}
\end{equation}

This is expected -- when $W_A = W_B = N$, $E[\eta(A, A \uplus B)] =
E[\eta(A, B)]$ since given $A$, $A \uplus B$ has the same distribution as
$B$.  $\eta$ distributes over the individual bits (which are
independently distributed) giving $E[\eta(A, B)] = N \times E[\eta(P,
  Q), P, Q \in \{0, 1\}] = \frac{N}{2}$.
 
