\documentclass[xcolor=dvipsnames]{beamer}
\usepackage{xspace}
\usepackage[style=authortitle-ibid,babel=hyphen,backend=biber]{biblatex}
\usepackage{minted}
\usecolortheme[named=Brown]{structure}
\usetheme[height=7mm]{Rochester}

\bibliography{../References}

\newcommand{\wattage}{\textsc{Wattage}\xspace}
\newminted{c}{linenos, frame=lines, baselinestretch=1}
\newminted{gas}{linenos, frame=lines, baselinestretch=1}

\title[Energy Optimization]{Optimizing Programs for Energy Efficiency}
%\subtitle[Errors]{}
\author[Sanjoy Das]{Sanjoy \textsc{Das}}
\institute[IITKGP]{
  Department of Mathematics\\
  Indian Institute of Technology Kharagpur\\
  India\\[1ex]
  \texttt{sanjoy@playingwithpointers.com}
}
\date[April 2013]{April 29, 2013}

\begin{document}

\begin{frame}[plain]
  \titlepage
\end{frame}

\begin{frame}{Overview}
  Energy consumption has become an important factor in system design.
  We intend to partially address this issue at the software level.

  \smallskip

  Our objectives are

  \begin{itemize}
    \item to be able to robustly estimate the power consumption of a
      program
    \item to analyze the addressable factors that affect this estimate
    \item to discuss techniques that improve this estimate
  \end{itemize}
\end{frame}

%% --------------------------------------------------------------------------

\begin{frame}{Energy Estimation via \wattage}
  In the first phase of the project we

  \begin{itemize}
    \item developed
      \wattage \footnote{https://github.com/sanjoy/wattage} -- a tool
      for estimating energy consumption of binaries (based on the
      model mentioned in \footcite{steinke}).
    \item estimated and analyzed the energy profiles of some common
      sorting algorithms implemented in C++.
  \end{itemize}
\end{frame}

\begin{frame}{Some Experimental Results : Sorting Algorithms}
  \includegraphics[width=1.0\textwidth]{../Figures/energy-vs-sort-algorithm.eps} 
\end{frame}

\begin{frame}{Some Experimental Results : Sorting Algorithms}
  \includegraphics[width=1.0\textwidth]
                  {../Figures/energy-per-inst-vs-sort-algorithm.eps}
\end{frame}

\begin{frame}{Some Experimental Results : Sorting Algorithms}
  \begin{figure}
    \centering
    \includegraphics[height=0.6\textheight]
                    {../Figures/energy-per-inst-nomem-vs-sort-algorithm.eps}
  \end{figure}

  \begin{equation}
    \phi = 1 - \frac{\text{Energy estimate without data address
        costs}}{\text{Total energy estimate}}
  \end{equation}
\end{frame}

\begin{frame}{Some Experimental Results : Quick Sort}
  \includegraphics[width=1.0\textwidth]{../Figures/qsort-energy-total.eps}
\end{frame}

\begin{frame}{Some Experimental Results : Quick Sort}
  \includegraphics[width=1.0\textwidth]{../Figures/qsort-energy-per-inst.eps}
\end{frame}

%% --------------------------------------------------------------------------

\begin{frame}{Optimizing Energy Profiles via llvm\footnote{http://llvm.org}}
  In the second phase of the project, we explored the possible
  compiler optimizations that may help improve the energy profiles of
  programs.
\end{frame}

\begin{frame}{Better Instruction Selection}
  \begin{itemize}
    \item Instruction selection has potential reduce
      \textit{per-instruction} energy costs.
    \item We changed llvm's code-generator-generator to respect
      instruction base costs.  \pause
    \item This approach did not work -- llvm generates the same
      assembly for the benchmarks before and after the augmentation.
    \item Two sequences of instructions semantically equivalent to
      each other but sufficiently different to have different energy
      profiles are rare.
  \end{itemize}
\end{frame}

\begin{frame}[fragile]{Better Instruction Scheduling}
  Consider the C function

  \begin{ccode}
    long access_memory(long *buf) {
      return buf[0] + buf[1] + buf[2] + buf[3] +
             buf[4] + buf[5] + buf[6] + buf[7];
    }
  \end{ccode}
\end{frame}

\begin{frame}[fragile]{Better Instruction Scheduling}
  Example assembly (hand-written, direct)

  \begin{gascode}
    access_memory:
      movq    (%rdi),   %rax
      addq    8(%rdi),  %rax
      addq    16(%rdi), %rax
      addq    24(%rdi), %rax
      addq    32(%rdi), %rax
      addq    40(%rdi), %rax
      addq    48(%rdi), %rax
      addq    56(%rdi), %rax
      ret
  \end{gascode}
\end{frame}

\begin{frame}[fragile]{Better Instruction Scheduling}
  Example assembly (hand-written, energy-optimized)

  \begin{gascode}
    access_memory:
      movq    (%rdi),   %rax
      addq    8(%rdi),  %rax
      addq    24(%rdi), %rax
      addq    16(%rdi), %rax
      addq    48(%rdi), %rax
      addq    56(%rdi), %rax
      addq    40(%rdi), %rax
      addq    32(%rdi), %rax
      ret
  \end{gascode}
\end{frame}

\begin{frame}{Better Instruction Scheduling}
  It \textit{is possible} to reduce energy costs associated with
  switching by intelligent scheduling.

  \smallskip

  We modified llvm's instruction scheduler to schedule

  \begin{itemize}
    \item similar instructions (same opcode) together
    \item memory access instructions trying to minimize switching
      costs in the memory address bus
    \item instructions with similar encodings together
  \end{itemize}

  whenever possible, given dependency constraints.
\end{frame}

\begin{frame}{Results: mildly successful}
  We saw

  \begin{itemize}
    \item no change in the energy profiles of the sorting algorithms.
    \item a small but consistent improvement in the energy profiles of
      some C++ compiler benchmarks from
      Adobe\footnote{http://stlab.adobe.com/performance/}
  \end{itemize}

\end{frame}

\begin{frame}{Results: mildly successful}
  \begin{figure}
    \centering
    \renewcommand{\arraystretch}{1.5}
    \begin{tabular}{| c | c | }
      \hline
      \textbf{Benchmark} & \textbf{\%-age improvement} \\
      \hline

      loop-unroll & 0.63\% \\
      stepanov-vector & 0.56\% \\
      stepanov-abstraction & 0.03\% \\
      \hline
    \end{tabular}
    \caption{Improvements in energy profile}
    \label{fig:improvements}
  \end{figure}
\end{frame}

\begin{frame}{Results: mildly successful}
  \begin{figure}[htbp]
    \centering
    \renewcommand{\arraystretch}{1.5}

    \begin{tabular}{| c | c |}
      \hline
      \textbf{Benchmark} & \textbf{\%-age decline} \\ \hline

      loop-unroll & -0.6 \% \\
      stepanov-vector & 0.1 \% \\
      stepanov-abstraction & 0.8 \% \\
      \hline
    \end{tabular}
    \caption{Decline in the runtime performance of the compiled
      executable in terms of CPU ticks (as reported by the
      \texttt{rdtsc} instruction)}
    \label{fig:performance-cost}
  \end{figure}
\end{frame}

\begin{frame}{Thank You}
  \begin{center}
    Thank You!\\
    Questions please
  \end{center}
\end{frame}

\begin{frame}{Bibliography}
  \printbibliography
\end{frame}

\end{document}
