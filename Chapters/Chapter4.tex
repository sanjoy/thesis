\chapter{Conclusions and Discussion}

\label{Chapter4}

\lhead{Chapter 4. \emph{Conclusions and Discussion}}

We started out with the goal of meaningfully understanding the energy
profile of software programs.  We saw in Chapter~\ref{Chapter2} that
an algorithmic feature that influences the energy profile of a program
is its data access patterns.  One can hope to reduce to see lower
energy consumption by making memory access as sequential as possible;
which gives us 2 bit flips in the memory address bus per fetch /
write, close to the optimal value of 1.  Haphazard memory access, on
the other hand, incurs more bit flips on the address bus; estimated as
$\frac{N}{2} + 1$ where $N$ is the bit width of the array index or its
equivalent (\refsec{eq:master-flip}).

We saw a limited amount of success in modifying llvm's instruction
scheduler to reduce the energy footprint of compiled binaries in
Chapter~\ref{Chapter3}.  While we don't think optimizing for energy
efficiency can be modelled as a vanilla data-flow problem; we think
there might be scope for dataflow \textit{analyses} that allow for
more intensive energy optimizations to take place later in the
compiler toolchain.  Returning to a brief version of an earlier
example, a compiler can usually not see that

\begin{gascode}
  movq    (%rdi),   %rax
  addq    8(%rdi),  %rax
  addq    24(%rdi), %rax
  addq    16(%rdi), %rax
\end{gascode}

and

\begin{gascode}
  movq    (%rdi),   %rax
  addq    8(%rdi),  %rax
  addq    16(%rdi), %rax
  addq    24(%rdi), %rax
\end{gascode}

are equivalent since it infers an output dependency on \texttt{\%rax}.
Ideally, though, a compiler should be able to use the commutativity of
addition and infer the equivalence of the last three instructions.
The energy aware scheduler should then be able to schedule the
instructions in the order of the first version; and incur lower
switching costs.
