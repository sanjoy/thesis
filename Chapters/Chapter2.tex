% Chapter Template

\chapter{Estimating power consumption with Pin}

\label{Chapter2}

\lhead{Chapter 2. \emph{Estimating power consumption with Pin}}

To estimate power consumption we implemented the estimation technique
mentioned in \cite[Steinke et. al.]{steinke} as a
pintool\footnote{https://github.com/sanjoy/wattage}.  We then used
this tool to estimate the power consumption of several sorting
algorithms.

Power consumed by a processor is split into four categories by
\cite{steinke} and the implementation reports each of them separately:

\begin{description*}
\item[$E_{cpu\_instr}$] the instruction-dependent costs inside the CPU
\item[$E_{cpu\_data}$] the data-dependent costs inside the CPU
\item[$E_{mem\_instr}$] the instruction-dependent costs in the
  instruction memory
\item[$E_{mem\_data}$] the data-dependent costs in the data memory
\end{description*}

\section{Factors contributing to power usage}

There are several processor traits or parameters highlighted by
\cite{steinke} as factors that contribute to the power profile of a
program.  The tool is parameterized on all these factors and the
significance of each one is listed in appendix \ref{AppendixA}.  Since
we did not have detailed information about x86 processors with us that
would allow us to infer these values, we had to resort to approximate
guesses for our experiments.  (TODO: document the exact values as an
appendix once they are finalized).

It is worth noting that this parameterization allows us to estimate or
simulate the power consumption of any arbitrary x86 micro-architecture
on any arbitrary x86 micro-architecture.  For instance, we can easily
simulate the power profile of an Intel Atom processor on an a Sandy
Bridge implementation of the x86 USA.

\section{The implementation}

\subsection{\texttt{wattage.so}: the pintool}

As mentioned above, the estimator is implemented as a pintool (which
will be referred to as \textit{wattage} from this point on) -- the
estimator frameowrk compiles to a shared object we pass to pin.  For
instance, a sample invocation that measures the non-IO energy
consumption for the POSIX utility \texttt{ls} will look like

\begin{minted}[frame=lines]{bash}
 $ pin -t wattage.so -output ls-power-consumption.out -traits_file \
   ...  x86.traits -- ls
  file-0   file-1   file-2   file-3   file-4
 $ cat ls-power-consumption.out
  [cpu_instr] = 9042445.000000
  [cpu_data] = 4538730.000000
  [mem_instr] = 6167330.000000
  [mem_data] = 514330.000000
  [TOTAL] = 20262834.000000
\end{minted}

The pintool registers a callback for \textit{each} instruction which
tracks the various contributing factors mentioned in \cite{steinke}.
Naturally, this is rather expensive; it isn't unusual for executables
to run 30,000 times slower under \textit{wattage}.  While there
certainly are opportunities for optimization, most of the performance
hit comes from unavoidable work; we need to compare the current,
possibly modified, register file with the previous register file at
instruction granularity.

\subsection{Gathering data from \texttt{tablegen}}

Apart from the factors mentioned in appendix \ref{AppendixA},
\textit{wattage} need to know the base cost for every instruction.
Providing this information per instruction is both tedious and
unnecessary -- dividing up the ISA into chunks of instructions with
similar itineraries allows us to provide a single base-cost figure per
such category.  Such a categorization can be extracted from the
database of architecture-specific information LLVM stores as
\texttt{*.td} files.

We used \texttt{tablegen}, a standard utility to parse and
semantically understand the architecture-specific \texttt{*.td} files
to generate tables that categorizes each x86 opcode to one of the 21
categories that \textit{wattage} knows about.  \texttt{tablegen} is
invoked as a part of \textit{wattage}'s build process.

\subsection{Non-determinism}

While the model we worked with is itself deterministic, there are
various issues which may make estimated power consumption differ from
run to run; \textit{even} in exceedingly simple programs that don't
have any obvious non-deterministic behavior.  This is due to the
interaction between the extremely low-level instrumentation of our
program and the
ASLR\footnote{http://en.wikipedia.org/wiki/Address\_space\_layout\_randomization}.

Since the \texttt{.text} and stack pages are likely to be mapped to
different virtual addresses each run, the instruction addresses and
the memory addresses are non-deterministic in a program that is
completely deterministic in the semantics of its source language.
This directly leads to the non-determinism of our power consumption
estimates.

\section{Experiments}


