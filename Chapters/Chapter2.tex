% Chapter Template

\chapter{Instruction Level Energy Modelling using Pin}

\label{Chapter2}

\lhead{Chapter 2. \emph{Instruction Level Energy Modelling using Pin}}

To estimate energy profiles of binaries we implemented the model
mentioned in \cite[Steinke et. al.]{steinke} as a
pintool\footnote{https://github.com/sanjoy/wattage}.  We then used
this tool (henceforth referred to as \wattage) to estimate and
analyze the energy profiles of several sorting algorithms.

Energy used by a processor is split into four categories by
\cite{steinke} and the implementation reports each of them separately:

\begin{description*}
\label{lst:fields-desc}
\item[$E_{cpu\_instr}$] the instruction-dependent costs inside the CPU
\item[$E_{cpu\_data}$] the data-dependent costs inside the CPU
\item[$E_{mem\_instr}$] the instruction-dependent costs in the
  instruction memory
\item[$E_{mem\_data}$] the data-dependent costs in the data memory
\end{description*}

\section{Processor Traits}

There are several processor traits or parameters highlighted by
\cite{steinke} as factors that contribute to the energy profile of a
program. \wattage is parameterized on all these factors and the
significance of each one is listed in \refsec{sec:parameters-desc}.
Since we did not have detailed information about x86
micro-architectures with us that would allow us to infer these values,
we had to resort to approximate guesses for our experiments.  The
values used in our experiments are listed in
\refsec{sec:trait-values}.

It is worth noting that this parameterization allows us to estimate or
simulate the energy consumption of any arbitrary x86
micro-architecture on any arbitrary x86 micro-architecture.  For
instance, we can easily simulate the energy profile of an Intel Atom
processor on an a Sandy Bridge implementation of the x86 ISA.

\section{Implementation}

\subsection{\texttt{wattage.so}: the pintool}

As mentioned above, \wattage is implemented as a pintool -- the
estimator framework compiles to a shared object we pass to pin.  For
instance, a sample invocation that measures the non-IO energy
consumption for the POSIX utility \texttt{ls} will look like

\begin{minted}[frame=lines]{bash}
 $ pin -t wattage.so -output ls-energy-consumption.out -traits_file \
   ...  x86.traits -- ls
  file-0   file-1   file-2   file-3   file-4
 $ cat ls-energy-consumption.out
  [INST_COUNT] = 648332
  [cpu_instr] = 2733213184.00
  [cpu_data] = 3309061376.00
  [mem_instr] = -741241280.00
  [mem_data] = -58891504.00
  [TOTAL] = 5.242142e+09
\end{minted}

\texttt{INST\_COUNT} is the total number of instructions actually
executed.  The next four fields are the four categories of energy
consumption, as mentioned in \refsec{lst:fields-desc}.  Interestingly,
in the above case, $E_{mem\_instr}$ and $E_{mem\_data}$ are estimated
to be negative.  This is a consequence of the fact that we don't have
estimates for per-instruction energy factors (and assume these to be
zero), while we \textit{do} have negative estimates for certain other
factors like \texttt{data\_addr\_mem\_wt\_individual}.  Unsettling as
this may be, since the total value of $E_{mem\_instr}$ and
$E_{mem\_data}$ are algebraic additions of relevant factors, negative
energy estimates are not an issue for comparative study.

\wattage works by registering a callback for \textit{each} instruction
which tracks the various contributing factors mentioned in
\cite{steinke}.  Naturally, this is rather expensive; it isn't unusual
for programs to run 30,000 times slower under \wattage.  While there
certainly are opportunities for optimization, most of the performance
hit comes from unavoidable work; for instance we need to compare the
current, possibly modified, register file with the previous register
file at instruction granularity.

\subsection{Gathering data from \texttt{tablegen}}
\label{sec:pintool-tablegen}

Apart from the factors mentioned in \refsec{AppendixA}, \wattage need
to know the base cost (referred to as \textit{BaseCPU} in
\cite{steinke}) for every x86 instruction.  Providing this information
per instruction is both tedious and unnecessary -- dividing up the ISA
into chunks of instructions with similar itineraries allowed us to
provide a single base-cost figure per such category.  Such a
categorization can be extracted from the database of
architecture-specific information LLVM stores as \texttt{*.td} files.

We used \texttt{tablegen}, a standard utility to parse and
semantically understand the architecture-specific \texttt{*.td} files,
to generate tables that categorizes each x86 opcode to one of the 21
categories that \wattage knows about.  \texttt{tablegen} is invoked as
a part of \wattage's build process.

\subsection{Non-determinism}

While the model we worked with is itself deterministic, the estimated
energy consumption may differ from run to run; \textit{even} in
exceedingly simple programs that don't have any obvious
non-deterministic behavior.  This is due to the interaction between
the extremely low-level instrumentation \wattage does and ASLR
(Address Space Layout Randomization)\footnote{ Address space layout
  randomization (ASLR) is a computer security method which involves
  randomly arranging the positions of key data areas, usually
  including the base of the executable and position of libraries,
  heap, and stack, in a process's address space.  For more details,
  see
  http://en.wikipedia.org/wiki/Address\_space\_layout\_randomization}.

Since the \texttt{.text} and stack pages are likely to be mapped to
different virtual addresses each run, the instruction and emmory
addresses are non-deterministic in programs that are completely
deterministic in the semantics of their source language.  This
directly leads to non-determinism in our estimates.

\section{Experiments}

To justify the viability of our tool, we ran some experiments in
measuring the energy characteristics of sample programs.

\subsection{Energy Profiles of Sorting Algorithms}

In this experiment, we estimated the energy profile of four common
sorting algorithms (the source code to which can be found in
\refsec{code:sorting-benchmark}):

\begin{itemize*}
\item Bubble Sort
\item Insertion Sort
\item Quick Sort
\item Merge Sort
\end{itemize*}

These sort algorithms were run on a pseudo-randomly generated (at
compile-time) array of a thousand integers and the energy estimates
were averaged over twenty runs.

\begin{figure}[htbp]
  \centering
  \includegraphics{Figures/energy-vs-sort-algorithm.eps}
  \rule{35em}{0.5pt}
  \caption{Total estimated energy consumption for some sorting algorithms}
  \label{fig:total-energy-sort-algo}
\end{figure}

The estimated total energy consumption, as shown in figure
\reffig{fig:total-energy-sort-algo}, is unsurprising in the sense that
the two most CPU efficient algorithms, quick sort and merge sort are
also the most energy efficient.  The additional energy requirement of
merge sort can be attributed to the writes to the auxiliary array.
Similarly the two least CPU efficient algorithms, bubble sort and
insertion sort, are also the most energy inefficient.  The
implementation of bubble sort used (see
\refsec{code:sorting-benchmark}) makes at least $\frac{n (n - 1)}{2}$
reads on any input and this it is slower than insertion sort on
average input -- explaining the additional energy consumed by bubble
sort over insertion sort.

The above graph, however, is dominated by the asymptotic efficiency of
the algorithm; a program that runs for a significantly longer period
of time \textit{will} consume more energy.  A more interesting metric
in this respect is the average energy consumed per instruction,
plotted in \reffig{fig:energy-per-inst-sort-algo}.

\begin{figure}[htbp]
  \centering
  \includegraphics{Figures/energy-per-inst-vs-sort-algorithm.eps}
  \rule{35em}{0.5pt}
  \caption{Estimated energy consumption per instruction for some sorting algorithms}
  \label{fig:energy-per-inst-sort-algo}
\end{figure}

Clearly, energy consumption doesn't just have to do with the running
time of the algorithm -- some programs also consume more
\textit{power}\footnote{Only in an approximate sense -- different
  instructions complete in different number of clock cycles.} than
others.  One aspect of this variation somewhat visible at the
algorithmic level are the data access patterns.  To quantify this, we
calculate

\begin{equation}
\phi = 1 - \frac{\text{Energy estimate without data address
    costs}}{\text{Total energy estimate}}
\end{equation}

\textit{Energy estimate without data address costs} can be calculated
by running \wattage with the \texttt{data\_addr\_*} traits to set 0.
$\phi$ plotted in \reffig{fig:energy-per-inst-nomem-sort-algo}.

\begin{figure}[htbp]
  \centering
  \includegraphics{Figures/energy-per-inst-nomem-vs-sort-algorithm.eps}
  \rule{35em}{0.5pt}
  \caption{Fraction of energy spent in the data bus ($\phi$)}
  \label{fig:energy-per-inst-nomem-sort-algo}
\end{figure}

Since the set of addresses accessed by each algorithm is the same, the
differences between $\phi$ for different algorithms should be
explainable with switching costs alone.  Partitioning in quick sort
and merging in merge sort both exhibit linear memory access behavior,
and consecutive addresses have a hamming distance of 2\footnote{For a
  proof, see \refsec{proof:average-2-flips}.}.  Both insertion sort
and bubble sort exhibit more haphazard memory access behaviour and
this pushes up the average switching cost from that in quick sort and
merge sort.  A plausible reason for insertion sort incurring higher
switching costs than bubble sort is the bubble sort algorithm spends
most of its time in linearly traversing through the input array.
Insertion sort, in the average case, incurs more \textit{jumps} than
bubble sort; bumping up this average hamming distance.  These jumps
allow insertion sort to finish earlier than bubble-sort (and consume
less \textit{total} energy, as seen in
\reffig{fig:total-energy-sort-algo}).

\subsection{Energy Characteristics of the Quick Sort Algorithm}

As a second experiment, we plotted the energy consumed to run the
quick sort algorithm with the size of the input array.  The total
energy consumed (\reffig{fig:qsort-energy-per-array-length}) is
dominated by quick sort's CPU time, as expected.

\begin{figure}[htbp]
  \centering
  \includegraphics{Figures/qsort-energy-total.eps}
  \rule{35em}{0.5pt}
  \caption{Estimated energy consumption by quicksort for a given array size}
  \label{fig:qsort-energy-per-array-length}
\end{figure}

We noted earlier that the energy consumed per instruction is more
amenable to analysis than the total energy consumed.  That is the case
here too.  We observe in figure
\reffig{fig:qsort-energy-per-inst-per-array-length} that the average
energy per instruction goes down as the array size increases.  This
can be explained using equation \refsec{eq:solved} -- the quick sort
algorithm spends more time accessing contiguous memory locations when
sorting large arrays than when sorting small ones.

\begin{figure}[htbp]
  \centering
  \includegraphics{Figures/qsort-energy-per-inst.eps}
  \rule{35em}{0.5pt}
  \caption{Estimated energy consumption per instruction by quicksort
    for a given array size}
  \label{fig:qsort-energy-per-inst-per-array-length}
\end{figure}

\section{Future Work}

While we'd like to think \wattage is currently very usable, there are
some issues that can be addressed

\begin{description}
\item[Real CPU Traits] \hfill \\ All the above analysis and profiling
  is based on processor traits \textit{derived} from published
  literature.  Such values may or may not be relevant in currently
  prevalent x86 microarchitectures.  One major step towards robustness
  would be to use regression on real data points to calculate the
  processor traits of some commercially available processors.

\item[Support for non-x86 ISAs] \hfill \\ Currently \wattage can only
  be used to analyze x86 programs.  It may be possible to extract
  architecture independent parts of the codebase and retarget it to
  other architectures like ARM.  Since pin only supports the x86 ISA,
  some other tool like Valgrind\footnote{http://valgrind.org/} will
  have to be used.

\item[Track Functional Units] \hfill \\ \wattage doesn't track
  functional unit changes.  Using detailed published information for
  specific x86 micro-architectures, it should be possible to extend
  \wattage with instruction itineraries and have it account for
  inter-instruction effects due to changes in the functional units
  recruited in the current instruction.

\item[Partial Profiling] \hfill \\ \wattage slows down the programs
  being profiled by a factor of 10,000 to 30,000.  An easy way to
  profile only certain user-specified functions might help ease the
  development workflow.

\item[Account for Cache Behaviour] \hfill \\ It should be possible to
  extend \wattage with a microarchitecture-appropriate cache-simulator
  and account for cache hits and misses in the energy profile.
\end{description}
